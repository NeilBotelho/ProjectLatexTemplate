\chapter{Introduction}
\section{Overview}
\subsection{Big data in media}
Media and entertainment has become an integrated part of the lives of people, meaning that people nowadays are very enthusiastic about trying new content in terms of watching it and choosing it. Gone are the single-channel days of no choice and no integration and consultation to viewers. But now these dynamics are changing, there are millions of watching options to choose from and they are also available to be streamed across various devices and are really getting user friendly.\newline

Publishers, broadcasters, news organizations, cable companies, and gaming companies in the media and entertainment industry are facing new business models for the way they create, market, and distribute their content. This is happening because today’s consumers search and access content anywhere, at any time, and on any device. As a result, there’s increased pressure to execute new digital production and multi-channel advertising and distribution strategies that rely on a detailed understanding of consumers’ media consumption preferences and behaviors. And, as consumer interests shift from analog to digital media, there are substantial opportunities to monetize content and to identify new products and services. With thousands or millions of digital consumers, media and entertainment companies are in a unique position to leverage their big data assets for more profitable customer engagement. \newline

The scope of big data collected by the media and entertainment industry and the potential to mine it to understand what content, shows, movies, and music consumers want is huge. Viewing history, searches, reviews, ratings, location and device data, clickstreams, log files, and social media sentiment are just a few data sources that help take the guesswork out of identifying audience interest. Using insights from big data, media and entertainment companies are able to understand when customers are most likely to view content and what device they’ll be using when they view it. With big data’s scalability, this information can be analyzed at a granular ZIP code level for localized distribution.

\subsection{AI in media industry}
The media and entertainment industry is also utilizing the power of Artificial Intelligence (AI) in making the visual content more interactive and interesting. It is helping to serve the audience a data-intensive and personalize automated content making their viewing experience more interesting and entertaining.

From music apps to OTT platforms, the audio, as well as visual contents, can be personalized as per the preferences and previous experiences of the user. Using machine learning, users behaviour and demographic details are taken into consideration to provide optimal suggestions for music or videos that they are likely to enjoy.

\subsection{Netflix test case}
Although netflix began as a meagre DVD rental service in 1998 it is now one of the world’s most powerful and renowned media streaming services. Having garnered subscribers up to almost 158.3 million, an estimate of nearly 37\% of the world’s internet users are using Netflix. A large portion of Netflix's growth can be attributed to its  personalized recommendation engine that enables users to find suggestion for new content, personalized to them.  Netflix's personalized recommendation engine is estimated to be worth  1 billion a year and surprisingly 80\% of users stream content suggested by it.

\subsection{Use of Data Science and algorithms to improve customer experience}
In the early 2000s, Netflix had initiated an open competition offering 1 million dollars prize for the best collective filtering algorithm to predict the ratings of users for films, based on their previous ratings. This approach resulted in becoming the turning point for the service.
Now, Netflix uses an opulence of technological algorithms to boost and enhance its customer experience.
In a service like Netflix, every action the user takes is recorded. The shows watched, the time of day when they are viewed, what was watched before and after that show, how quickly a series is binge-watched, when and where a user gets bored and stops watching, how long does a user take to scroll and every single click of the pause and play button. Using a detailed tagging system Netflix is able to recommend it’s users the content which it knows will be their cup of tea.
Recommendation Systems are mainly of two main types :
Content-based Recommendation Systems: In this system, the background knowledge regarding the products as well as the customer information is taken into account. Similar suggestions are provided based on the content the user has viewed on Netflix. For example, if the user has watched a film that has a "thriller" genre, similar films, having the same genre will be suggested.
Collaborative filtering Recommendation Systems: This system provides suggestions based on the similar profiles of its users and is independent of knowledge of the product. This system is based solely on the assumption that what the users prefer in the past they will also prefer in the future.

\section{Motivation}
The outbreak of the Coronavirus (COVID-19) pandemic and its preventative social distancing measures have led to a dramatic increase in subscriptions to paid streaming services. Online users are increasingly accessing live broadcasts, as well as recorded video content and digital music services through internet and mobile devices. In this context, this study aims to explore the individuals’ uses and gratifications from online streaming technologies during COVID-19.
Individuals are increasingly consuming the broadcast media through digital and mobile technologies. Very often, they are watching TV channels, movies, series, shows, etc. through online streaming services that are readily available through ubiquitous technologies, including smartphones or tablets. 
eMarketer (2019) reported that 70.1\% surfed the internet while watching their favorite movies and shows. Moreover, according to the latest Global Web Index Trend Report, the individuals who were between 16–24 years, spent around 7h per day online or on their smartphones or tablets. The individuals from this demographic segment devoted over 2.5 h a day to social networking and were watching more than an hour of online TV per day.\newline

The individuals hailing from the 25–34 age segment have switched from linear TV to online streaming to watch live TV and/or recorded videos. They subscribed to online services through
digital and high-speed mobile devices, including smartphones and tablets to stream live channels and recorded video content from anywhere, at any time (eMarketer, 2019; GWI, 2019). Evidently, they were accessing online streaming through virtual private networks to watch TV programs, movies, entertainment, sporting events and the like (GWI, 2019). Hence, media and entertainment businesses are continuously investing on the programming of new content, including those produced in-house to satisfy their online subscribers. In this light, this study explores the individuals’ perceptions toward online streaming technologies and their motivations to use them to watch recorded videos and/or live broadcasts.\newline

In the age of streaming and on-demand media, there is an overabundance of content for users to consume.
According to FX Networks, in 2019 approximately 532 new scripted tv shows were published.
Spotify reports that over 60,000 new songs are uploaded to its service every day.
Hence discoverability of new media that is interesting to a user is a massive issue. \newline

The current solution is to use opaque algorithms to predict what the user may like. In this project we propose developing a more transparent alternative. 
Due to the addition of a large amount of media everyday, it is the need of hour to make content discovery easy and transparent for the users. 


